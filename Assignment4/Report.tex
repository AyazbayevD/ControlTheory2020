%%% Template originaly created by Karol Kozioł (mail@karol-koziol.net) and modified for ShareLaTeX use

\documentclass[a4paper,11pt]{article}

\usepackage[T1]{fontenc}
\usepackage[utf8]{inputenc}
\usepackage{graphicx}
\usepackage{hyperref}
\usepackage{xcolor}

\renewcommand\familydefault{\sfdefault}
\usepackage{tgheros}
\usepackage[defaultmono]{droidmono}

\usepackage{amsmath,amssymb,amsthm,textcomp}
\usepackage{enumerate}
\usepackage{multicol}
\usepackage{tikz}

\usepackage{geometry}
\geometry{left=25mm,right=25mm,%
bindingoffset=0mm, top=20mm,bottom=20mm}


\linespread{1.3}

\newcommand{\linia}{\rule{\linewidth}{0.5pt}}

% custom theorems if needed
\newtheoremstyle{mytheor}
    {1ex}{1ex}{\normalfont}{0pt}{\scshape}{.}{1ex}
    {{\thmname{#1 }}{\thmnumber{#2}}{\thmnote{ (#3)}}}

\theoremstyle{mytheor}
\newtheorem{defi}{Definition}

% my own titles
\makeatletter
\renewcommand{\maketitle}{
\begin{center}
\vspace{2ex}
{\huge \textsc{\@title}}
\vspace{1ex}
\\
\linia\\
\@author \hfill \@date
\vspace{4ex}
\end{center}
}
\makeatother
%%%

% custom footers and headers
\usepackage{fancyhdr}
\pagestyle{fancy}
\lhead{}
\chead{}
\rhead{}
\cfoot{}
\rfoot{Page \thepage}
\renewcommand{\headrulewidth}{0pt}
\renewcommand{\footrulewidth}{0pt}
%

% code listing settings
\usepackage{listings}
\lstset{
    language=Python,
    basicstyle=\ttfamily\small,
    aboveskip={1.0\baselineskip},
    belowskip={1.0\baselineskip},
    columns=fixed,
    extendedchars=true,
    breaklines=true,
    tabsize=4,
    prebreak=\raisebox{0ex}[0ex][0ex]{\ensuremath{\hookleftarrow}},
    frame=lines,
    showtabs=false,
    showspaces=false,
    showstringspaces=false,
    keywordstyle=\color[rgb]{0.627,0.126,0.941},
    commentstyle=\color[rgb]{0.133,0.545,0.133},
    stringstyle=\color[rgb]{01,0,0},
    numbers=left,
    numberstyle=\small,
    stepnumber=1,
    numbersep=10pt,
    captionpos=t,
    escapeinside={\%*}{*)}
}

%%%----------%%%----------%%%----------%%%----------%%%

\begin{document}

\title{Control Theory Assignment \textnumero{} 4}

\author{Danat Ayazbayev, BS-18-03}

\maketitle

\section*{Selecting variant.}
name: Danat\newline
email: d.ayazbaev@innopolis.university\newline
variant: e

\section*{Consider classical benchmark system in control theory - inverted pendulum on a cart. It is nonlinear under-actuated system that has the following dynamics:}
\[
(M+m)\ddot{x}-mlcos(\theta)\ddot{\theta}+mlsin(\theta)\dot{\theta}^{2}=F
\]
\[
-cos(\theta)\ddot{x}+l\ddot{\theta}-gsin(\theta)=0
\]
\[
g=9.81, M=7.5, m=4.4, l=1.2
\]
\[
\includegraphics[]{CTHW4givenpic.JPG}
\]
A schematic drawing of the inverted pendulum on a cart. The rod is considered massless. The mass of the cart and the point mass at the end of the rod are denoted by $M$ and $m$. The rod has a length $l$.
\subsection*{(A)} 
\textbf{Write equations of motion of the system in manipulator form} \newline
\[
M(q)\ddot{q}+n(q,\dot{q})=Bu
\]
\textbf{where} $u = F$, $q = [$ $x$\space\space$\theta$ $]^{T}$ \textbf{is vector of generalized coordinates.}\newline
Referring to the material by this           
\href{https://ocw.mit.edu/courses/electrical-engineering-and-computer-science/6-832-underactuated-robotics-spring-2009/readings/MIT6_832s09_read_ch03.pdf}{\textcolor{blue}{\underline{link}}}\newline
\[
    H(q)\ddot{q}+C(q, \dot{q})\dot{q}+G(q)=Bu
\]
where
\[
H(q)=
\begin{bmatrix}
M+m&mlcos(\theta+\pi)\\
mlcos(\theta+\pi)&ml^{2}\\
\end{bmatrix}
,
C(q, \dot{q})=
\begin{bmatrix}
0&-ml\dot{\theta} sin(\theta+\pi)\\
0&0\\
\end{bmatrix},
\]
\[
G(q)=
\begin{bmatrix}
0\\
mglsin(\theta+\pi)
\end{bmatrix}
,
B=
\begin{bmatrix}
1\\
0\\
\end{bmatrix}
\]
\textbf{Note:} $\pi$ is added to the $\theta$ inside trigonometric functions because in the material by the link angle $\theta$ is defined not as in the assignment.\newline 

\subsection*{(B)} 
\textbf{Write dynamics of the system in control affine nonlinear form} \newline
\[
\dot{z}=f(z)+g(z)u
\]
\textbf{where} $z = [$ $x$\space\space$\theta$\space\space$\dot{x}$\space\space$\dot{\theta}$ $]^{T}$ \textbf{is vector states of the system.}\newline
Let's extract $\ddot{x}$ and $\ddot{\theta}$ from these equations:
\[
(M+m)\ddot{x}-mlcos(\theta)\ddot{\theta}+mlsin(\theta)\dot{\theta}^{2}=F
\]
\[
-cos(\theta)\ddot{x}+l\ddot{\theta}-gsin(\theta)=0
\]
\[
\ddot{x}=\frac{F-mlsin(\theta)\dot{\theta}^{2}+mlcos(\theta)\ddot{\theta}}{M+m}
\]
\[
\ddot{\theta}=\frac{gsin(\theta)+cos(\theta)\ddot{x}}{l}
\]
\[
\ddot{x}=\frac{F-mlsin(\theta)\dot{\theta}^{2}+mlcos(\theta)(\frac{gsin(\theta)+cos(\theta)\ddot{x}}{l})}{M+m}
\]
\[
\ddot{x}=\frac{Fl-ml^{2}sin(\theta)\dot{\theta}^{2}+mlgscos(\theta)sin(\theta)+mlcos^{2}(\theta)\ddot{x}}{(M+m)l}
\]
\[
(M+m)l\ddot{x}-mlcos^{2}(\theta)\ddot{x}=Fl-ml^{2}sin(\theta)\dot{\theta}^{2}+mlgcos(\theta)sin(\theta)
\]
Here's $\ddot{x}$
\[
\ddot{x}=\frac{F-mlsin(\theta)\dot{\theta}^{2}+mgcos(\theta)sin(\theta)}{M+m-mcos^{2}(\theta)}
\]
Now let's find $\ddot{\theta}$
\[
\ddot{\theta}=\frac{gsin(\theta)+cos(\theta)(\frac{F-mlsin(\theta)\dot{\theta}^{2}+mgcos(\theta)sin(\theta)}{M+m-mcos^{2}(\theta)})}{l}
\]
\[
\ddot{\theta}=\frac{Mlgsin(\theta)+mlgsin(\theta)-mlgsin(\theta)cos^{2}(\theta)+Fcos(\theta)-mlsin(\theta)\dot{\theta}^{2}cos(\theta)+mgsin(\theta)cos^{2}(\theta)}{Ml+ml-mlcos^{2}(\theta)}
\]
Now let's write system dynamics in control affine nonlinear form
\[
\begin{bmatrix}
\dot{x}\\
\dot{\theta}\\
\ddot{x}\\
\ddot{\theta}\\
\end{bmatrix}
=
\begin{bmatrix}
\dot{x}\\
\dot{\theta}\\
\frac{-mlsin(\theta)\dot{\theta}^{2}+mgcos(\theta)sin(\theta)}{M+m-mcos^{2}(\theta)}\\
\frac{Mlgsin(\theta)+mlgsin(\theta)-mlgsin(\theta)cos^{2}(\theta)-mlsin(\theta)\dot{\theta}^{2}cos(\theta)+mgsin(\theta)cos^{2}(\theta)}{Ml+ml-mlcos^{2}(\theta)}\\
\end{bmatrix}
+
\begin{bmatrix}
0\\
0\\
\frac{1}{M+m-mcos^{2}(\theta)}\\
\frac{cos(\theta)}{Ml+ml-mlcos^{2}(\theta)}\\
\end{bmatrix}
u
\]
\[
\begin{bmatrix}
\dot{x}\\
\dot{\theta}\\
\ddot{x}\\
\ddot{\theta}\\
\end{bmatrix}
=
\begin{bmatrix}
\dot{x}\\
\dot{\theta}\\
\frac{-5.28sin(\theta)\dot{\theta}^{2}+43.164cos(\theta)sin(\theta)}{11.9-4.4cos^{2}(\theta)}\\
\frac{88.29sin(\theta)+47.088sin(\theta)-47.088sin(\theta)cos^{2}(\theta)-5.28sin(\theta)\dot{\theta}^{2}cos(\theta)+43.12sin(\theta)cos^{2}(\theta)}{3.72-5.28cos^{2}(\theta)}\\
\end{bmatrix}
+
\begin{bmatrix}
0\\
0\\
\frac{1}{11.9-4.4cos^{2}(\theta)}\\
\frac{cos(\theta)}{3.72-5.28cos^{2}(\theta)}\\
\end{bmatrix}
u
\]
\subsection*{(C)} 
\textbf{Linearize nonlinear dynamics of the systems around equilibrium point} $\overline{z} = [$ $0$\space\space$0$\space\space$0$\space\space$0$ $]^{T}$\newline
\[
\delta\dot{z}=A\delta z+B\delta u
\]
Referring to the material by this           
\href{http://leo.technion.ac.il/Courses/LS/lect03.pdf}{\textcolor{blue}{\underline{link}}}\newline
\[
A=
\begin{bmatrix}
0&0&1&0\\
0&0&0&1\\
0&\frac{mg}{M}&0&0\\
0&\frac{(m+M)g}{Ml}&0&0\\
\end{bmatrix}
,B=
\begin{bmatrix}
0\\
0\\
\frac{1}{M}\\
\frac{1}{Ml}\\
\end{bmatrix}
\]
\[
A=
\begin{bmatrix}
0&0&1&0\\
0&0&0&1\\
0&\frac{3597}{625}&0&0\\
0&\frac{119}{75}&0&0\\
\end{bmatrix}
,B=
\begin{bmatrix}
0\\
0\\
\frac{2}{15}\\
\frac{1}{9}\\
\end{bmatrix}
\]
\textbf{Note}\newline
Friction coefficient is zero in the task, so $A_{3,3}, A_{4,3}$ both are $0$.\newline
Also equations are constructed differently in the material, and for this task correct thing will be positive signs at $A_{3,2}$ and $B_{4,1}$
\subsection*{(D)}
\textbf{Check stability of the linearized system using any method.}\newline
Let's find eigenvalues of matrix
\[
A=
\begin{bmatrix}
0&0&1&0\\
0&0&0&1\\
0&\frac{mg}{M}&0&0\\
0&\frac{(m+M)g}{Ml}&0&0\\
\end{bmatrix}
=
\begin{bmatrix}
0&0&1&0\\
0&0&0&1\\
0&\frac{3597}{625}&0&0\\
0&\frac{119}{75}&0&0\\
\end{bmatrix}
\]
\[
eig(A)=>
\lambda_{1}=\frac{\sqrt{357}}{15}, \lambda_{2}=-\frac{\sqrt{357}}{15}, \lambda_{3,4}=0
\]
System is unstable, because not all eigenvalues less than $0$.
\subsection*{(E)}
\textbf{Check if linearized system is controllable; if not - try another variant
or change values of your variant and find controllable.}\newline
Let
\[
\Gamma=
\begin{bmatrix}
B&AB&A^{2}B&A^{3}B\\
\end{bmatrix}
\]
\[
A=
\begin{bmatrix}
0&0&1&0\\
0&0&0&1\\
0&\frac{mg}{M}&0&0\\
0&\frac{(m+M)g}{Ml}&0&0\\
\end{bmatrix}
,B=
\begin{bmatrix}
0\\
0\\
\frac{1}{M}\\
\frac{1}{Ml}\\
\end{bmatrix}
\]
\[
AB=
\begin{bmatrix}
\frac{1}{M}\\
\frac{1}{Ml}\\
0\\
0\\
\end{bmatrix}
,A^{2}B=
\begin{bmatrix}
0\\
0\\
\frac{mg}{M^{2}l}\\
\frac{m+M}{M^{2}l^{2}}\\
\end{bmatrix}
,A^{3}B=
\begin{bmatrix}
\frac{mg}{M^{2}l}\\
\frac{m+M}{M^{2}l^{2}}\\
0\\
0\\
\end{bmatrix}
\]
\[
\Gamma=
\begin{bmatrix}
0&\frac{1}{M}&0&\frac{mg}{M^{2}l}\\
0&\frac{1}{Ml}&0&\frac{m+M}{M^{2}l^{2}}\\
\frac{1}{M}&0&\frac{mg}{M^{2}l}&0\\
\frac{1}{Ml}&0&\frac{m+M}{M^{2}l^{2}}&0\\
\end{bmatrix}
=
\begin{bmatrix}
0&\frac{2}{15}&0&\frac{1199}{1875}\\
0&\frac{1}{9}&0&\frac{119}{810}\\
\frac{2}{15}&0&\frac{1199}{1875}&0\\
\frac{1}{9}&0&\frac{119}{810}&0\\
\end{bmatrix}
\]
\[
rank(\Gamma)=4
\]
The linearized system is controllable, because $\Gamma$ has full rank.
\subsection*{(F)}
\textbf{(for the controllable system) Design state feedback controller for linearized system using pole placement method. Assess the performance
of the controller for variety of initial conditions. Justify the choice of
initial conditions. Solve the task by two ways: using root-locus and
with python. Compare them.}
\subsection*{(G)}
\textbf{(for the controllable system) Design linear quadratic regulator for linearized system. Assess the performance of the controller for variety
of initial conditions. Justify the choice of initial conditions.}
\end{document}